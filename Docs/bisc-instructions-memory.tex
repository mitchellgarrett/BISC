\subsection{Memory Instructions}

\subsubsection{LDW \normalfont{- Load Word}}\label{sec:LDW}

\vspace{3ex}

\begin{center}
	\begin{bytefield}[leftcurly=., leftcurlyspace=0pt]{32}
		\bitheader[endianness=little]{0, 7, 8, 15, 16, 23, 24, 31} \\
		\begin{leftwordgroup}{\hyperref[sec:m-type]{\textbf{M-Type}}}
		\bitbox{8}{0x08} & 
		\bitbox{8}{rd} &
		\bitbox{8}{rs}
		\bitbox{8}{offset}
		\end{leftwordgroup}\\
	\end{bytefield}
\end{center}

\textbf{Description}

The LDW instruction loads the 32-bit word from memory at the address given by the value of the source register plus the offeset into the destination register.
This instruction is used to load values from memory into a register.
Note that the offset is a signed 8-bit value, so you can only index address +127 or -127 of the value of the source register.
Words are 4-byte values so LDW should only be used on addresses that are multiples of 4.

\vspace{3ex}

\textbf{Action}
\begin{lstlisting}[frame=single]
	rd := mem[rs + offset]
\end{lstlisting}

\vspace{3ex}

\textbf{Assembler Syntax}
\begin{lstlisting}[frame=single]
	LDW {rd}, {rs}[imm8]
\end{lstlisting}

\vspace{3ex}

\textbf{Example}
\begin{lstlisting}[frame=single]
	; load an address into r0
	LDI r0, THIS_IS_AN_ADDRESS
	
	; load the word at that address into r1
	LDW r1, r0[0]
	
	; load the next word into r2
	; by adding an offset of 4
	LDW r2, r0[4]
\end{lstlisting}

\subsubsection{LDH \normalfont{- Load Half Word}}\label{sec:LDH}

\vspace{3ex}

\begin{center}
	\begin{bytefield}[leftcurly=., leftcurlyspace=0pt]{32}
		\bitheader[endianness=little]{0, 7, 8, 15, 16, 23, 24, 31} \\
		\begin{leftwordgroup}{\hyperref[sec:m-type]{\textbf{M-Type}}}
		\bitbox{8}{0x09} & 
		\bitbox{8}{rd} &
		\bitbox{8}{rs}
		\bitbox{8}{offset}
		\end{leftwordgroup}\\
	\end{bytefield}
\end{center}

\textbf{Description}

The LDH instruction loads the 16-bit half word from memory at the address given by the value of the source register plus the offeset into the destination register.
The half word is placed into the lower 16 bits of the destination register. The upper 16 bits are zeroed.
Note that the offset is a signed 8-bit value, so you can only index address +127 or -127 of the value of the source register.
Half words are 2-byte values so LDH should only be used on addresses that are multiples of 2.

\vspace{3ex}

\textbf{Action}
\begin{lstlisting}[frame=single]
	rd[0..15] := mem[rs + offset]
\end{lstlisting}

\vspace{3ex}

\textbf{Assembler Syntax}
\begin{lstlisting}[frame=single]
	LDH {rd}, {rs}[imm8]
\end{lstlisting}

\vspace{3ex}

\textbf{Example}
\begin{lstlisting}[frame=single]
	; load an address into r0
	LDI r0, THIS_IS_AN_ADDRESS
	
	; load the half word at that address into r1
	LDH r1, r0[0]
	
	; load the next half word into r2
	; by adding an offset of 2
	LDH r2, r0[2]
\end{lstlisting}

\subsubsection{LDB \normalfont{- Load Byte}}\label{sec:LDB}

\vspace{3ex}

\begin{center}
	\begin{bytefield}[leftcurly=., leftcurlyspace=0pt]{32}
		\bitheader[endianness=little]{0, 7, 8, 15, 16, 23, 24, 31} \\
		\begin{leftwordgroup}{\hyperref[sec:m-type]{\textbf{M-Type}}}
		\bitbox{8}{0x0A} & 
		\bitbox{8}{rd} &
		\bitbox{8}{rs}
		\bitbox{8}{offset}
		\end{leftwordgroup}\\
	\end{bytefield}
\end{center}

\textbf{Description}

The LDB instruction loads the byte from memory at the address given by the value of the source register plus the offeset into the destination register.
The byte is placed into the lower 8 bits of the destination register. The upper 24 bits are zeroed.
Note that the offset is a signed 8-bit value, so you can only index address +127 or -127 of the value of the source register.

\vspace{3ex}

\textbf{Action}
\begin{lstlisting}[frame=single]
	rd[0..7] := mem[rs + offset]
\end{lstlisting}

\vspace{3ex}

\textbf{Assembler Syntax}
\begin{lstlisting}[frame=single]
	LDB {rd}, {rs}[imm8]
\end{lstlisting}

\vspace{3ex}

\textbf{Example}
\begin{lstlisting}[frame=single]
	; load an address into r0
	LDI r0, THIS_IS_AN_ADDRESS
	
	; load the byte at that address into r1
	LDB r1, r0[0]
	
	; load the next byte into r2
	; by adding an offset of 1
	LDB r2, r0[1]
\end{lstlisting}

\subsubsection{STW \normalfont{- Store Word}}\label{sec:STW}

\vspace{3ex}

\begin{center}
	\begin{bytefield}[leftcurly=., leftcurlyspace=0pt]{32}
		\bitheader[endianness=little]{0, 7, 8, 15, 16, 23, 24, 31} \\
		\begin{leftwordgroup}{\hyperref[sec:m-type]{\textbf{M-Type}}}
		\bitbox{8}{0x0B} & 
		\bitbox{8}{rd} &
		\bitbox{8}{rs}
		\bitbox{8}{offset}
		\end{leftwordgroup}\\
	\end{bytefield}
\end{center}

\textbf{Description}

The STW instruction stores the value of the source register into memory at the address given by the value of the destination register plus the offeset.
This instruction places an entire 4-byte value into memory.
Note that the offset is a signed 8-bit value, so you can only index address +127 or -127 of the value of the source register.

\vspace{3ex}

\textbf{Action}
\begin{lstlisting}[frame=single]
	mem[rd + offset] := rs
\end{lstlisting}

\vspace{3ex}

\textbf{Assembler Syntax}
\begin{lstlisting}[frame=single]
	LDW {rs}, {rd}[imm8]
\end{lstlisting}

\vspace{3ex}

\textbf{Example}
\begin{lstlisting}[frame=single]
	; load an address into r0
	LDI r0, THIS_IS_AN_ADDRESS
	
	; store the word in r1 into memory
	STW r1, r0[0]
	
	; store the word in r2 into the next word in
	; memory by adding an offset of 4
	STW r2, r0[4]
\end{lstlisting}

\subsubsection{STH \normalfont{- Store Half Word}}\label{sec:STH}

\vspace{3ex}

\begin{center}
	\begin{bytefield}[leftcurly=., leftcurlyspace=0pt]{32}
		\bitheader[endianness=little]{0, 7, 8, 15, 16, 23, 24, 31} \\
		\begin{leftwordgroup}{\hyperref[sec:m-type]{\textbf{M-Type}}}
		\bitbox{8}{0x0C} & 
		\bitbox{8}{rd} &
		\bitbox{8}{rs}
		\bitbox{8}{offset}
		\end{leftwordgroup}\\
	\end{bytefield}
\end{center}

\textbf{Description}

The STH instruction stores the lower 16 bits of the source register into memory at the address given by the value of the destination register plus the offeset.
This instruction only modifies the half word indexed by the address, no other bytes are modified.
Since STH stores a half word, the top 16 bits of the source register are ignored.
Note that the offset is a signed 8-bit value, so you can only index address +127 or -127 of the value of the source register.

\vspace{3ex}

\textbf{Action}
\begin{lstlisting}[frame=single]
	mem[rd + offset] := rs[0..15]
\end{lstlisting}

\vspace{3ex}

\textbf{Assembler Syntax}
\begin{lstlisting}[frame=single]
	LDH {rs}, {rd}[imm8]
\end{lstlisting}

\vspace{3ex}

\textbf{Example}
\begin{lstlisting}[frame=single]
	; load an address into r0
	LDI r0, THIS_IS_AN_ADDRESS
	
	; store the half word in r1 into memory
	STH r1, r0[0]
	
	; store the half word in r2 into the next
	; half word in memory by adding an offset of 2
	STH r2, r0[2]
\end{lstlisting}

\subsubsection{STB \normalfont{- Store Byte}}\label{sec:STB}

\vspace{3ex}

\begin{center}
	\begin{bytefield}[leftcurly=., leftcurlyspace=0pt]{32}
		\bitheader[endianness=little]{0, 7, 8, 15, 16, 23, 24, 31} \\
		\begin{leftwordgroup}{\hyperref[sec:m-type]{\textbf{M-Type}}}
		\bitbox{8}{0x0D} & 
		\bitbox{8}{rd} &
		\bitbox{8}{rs}
		\bitbox{8}{offset}
		\end{leftwordgroup}\\
	\end{bytefield}
\end{center}

\textbf{Description}

The STB instruction stores the lower 8 bits of the source register into memory at the address given by the value of the destination register plus the offeset.
This instruction only modifies the byte indexed by the address, no other bytes are modified.
Since STB only stores a single byte, the top 24 bits of the source register are ignored.
Note that the offset is a signed 8-bit value, so you can only index address +127 or -127 of the value of the source register.

\vspace{3ex}

\textbf{Action}
\begin{lstlisting}[frame=single]
	mem[rd + offset] := rs[0..7]
\end{lstlisting}

\vspace{3ex}

\textbf{Assembler Syntax}
\begin{lstlisting}[frame=single]
	LDB {rs}, {rd}[imm8]
\end{lstlisting}

\vspace{3ex}

\textbf{Example}
\begin{lstlisting}[frame=single]
	; load an address into r0
	LDI r0, THIS_IS_AN_ADDRESS
	
	; store the byte in r1 into memory
	STB r1, r0[0]
	
	; store the byte in r2 into the next byte in
	; memory by adding an offset of 1
	STB r2, r0[1]
\end{lstlisting}