\subsection{Bitwise Instructions}

\subsubsection{AND \normalfont{- Bitwise And}}\label{sec:AND}

\vspace{3ex}

\begin{center}
	\begin{bytefield}[leftcurly=., leftcurlyspace=0pt]{32}
		\bitheader[endianness=little]{0, 7, 8, 15, 16, 23, 24, 31} \\
		\begin{leftwordgroup}{}
			\bitbox{8}{0x1A} & 
			\bitbox{8}{rd} &
			\bitbox{8}{rs\textsubscript{0}}
			\bitbox{8}{rs\textsubscript{1}}
		\end{leftwordgroup}\\
	\end{bytefield}
\end{center}

\textbf{Description}

The AND performs a bitwise and operation on all of the bits in the two source registers and stores the result in the destination register.

\vspace{3ex}

\textbf{Action}
\begin{lstlisting}[frame=single]
	rd := rs $\land$ rs1
\end{lstlisting}

\vspace{3ex}

\textbf{Assembler Syntax}
\begin{lstlisting}[frame=single]
	AND {rd}, {rs0}, {rs1}
\end{lstlisting}

\vspace{3ex}

\textbf{Example}
\begin{lstlisting}[frame=single]
	AND r0, r1, r2
\end{lstlisting}

\subsubsection{OR \normalfont{- Bitwise Or}}\label{sec:OR}
 
\vspace{3ex}

\begin{center}
	\begin{bytefield}[leftcurly=., leftcurlyspace=0pt]{32}
		\bitheader[endianness=little]{0, 7, 8, 15, 16, 23, 24, 31} \\
		\begin{leftwordgroup}{}
			\bitbox{8}{0x1B} & 
			\bitbox{8}{rd} &
			\bitbox{8}{rs\textsubscript{0}}
			\bitbox{8}{rs\textsubscript{1}}
		\end{leftwordgroup}\\
	\end{bytefield}
\end{center}

\textbf{Description}

The OR performs a bitwise or operation on all of the bits in the two source registers and stores the result in the destination register.

\vspace{3ex}

\textbf{Action}
\begin{lstlisting}[frame=single]
	rd := rs $\lor$ rs1
\end{lstlisting}

\vspace{3ex}

\textbf{Assembler Syntax}
\begin{lstlisting}[frame=single]
	OR {rd}, {rs0}, {rs1}
\end{lstlisting}

\vspace{3ex}

\textbf{Example}
\begin{lstlisting}[frame=single]
	OR r0, r1, r2
\end{lstlisting}

\subsubsection{XOR \normalfont{- Bitwise Exlcusive Or}}\label{sec:XOR}

\vspace{3ex}

\begin{center}
	\begin{bytefield}[leftcurly=., leftcurlyspace=0pt]{32}
		\bitheader[endianness=little]{0, 7, 8, 15, 16, 23, 24, 31} \\
		\begin{leftwordgroup}{}
			\bitbox{8}{0x1C} & 
			\bitbox{8}{rd} &
			\bitbox{8}{rs\textsubscript{0}}
			\bitbox{8}{rs\textsubscript{1}}
		\end{leftwordgroup}\\
	\end{bytefield}
\end{center}

\textbf{Description}

The XOR performs a bitwise exclusive or operation on all of the bits in the two source registers and stores the result in the destination register.

\vspace{3ex}

\textbf{Action}
\begin{lstlisting}[frame=single]
	rd := rs $\oplus$ rs1
\end{lstlisting}

\vspace{3ex}

\textbf{Assembler Syntax}
\begin{lstlisting}[frame=single]
	XOR {rd}, {rs0}, {rs1}
\end{lstlisting}

\vspace{3ex}

\textbf{Example}
\begin{lstlisting}[frame=single]
	XOR r0, r1, r2
\end{lstlisting}

\subsubsection{BSL \normalfont{- Bit Shift Left}}\label{sec:BSL}

\vspace{3ex}

\begin{center}
	\begin{bytefield}[leftcurly=., leftcurlyspace=0pt]{32}
		\bitheader[endianness=little]{0, 7, 8, 15, 16, 23, 24, 31} \\
		\begin{leftwordgroup}{}
			\bitbox{8}{0x1D} & 
			\bitbox{8}{rd} &
			\bitbox{8}{rs\textsubscript{0}}
			\bitbox{8}{rs\textsubscript{1}}
		\end{leftwordgroup}\\
	\end{bytefield}
\end{center}

\textbf{Description}

The BSL instruction shifts the value of \texttt{rs\textsubscript{0}} left by \texttt{rs\textsubscript{1}} bits.

\vspace{3ex}

\textbf{Action}
\begin{lstlisting}[frame=single]
	rd := rs << rs1
\end{lstlisting}

\vspace{3ex}

\textbf{Assembler Syntax}
\begin{lstlisting}[frame=single]
	BSL {rd}, {rs0}, {rs1}
\end{lstlisting}

\vspace{3ex}

\textbf{Example}
\begin{lstlisting}[frame=single]
	BSL r0, r1, r2
\end{lstlisting}

\subsubsection{BSR \normalfont{- Bit Shift Right}}\label{sec:BSR}

\vspace{3ex}

\begin{center}
	\begin{bytefield}[leftcurly=., leftcurlyspace=0pt]{32}
		\bitheader[endianness=little]{0, 7, 8, 15, 16, 23, 24, 31} \\
		\begin{leftwordgroup}{}
			\bitbox{8}{0x1E} & 
			\bitbox{8}{rd} &
			\bitbox{8}{rs\textsubscript{0}}
			\bitbox{8}{rs\textsubscript{1}}
		\end{leftwordgroup}\\
	\end{bytefield}
\end{center}

\textbf{Description}

The BSR instruction shifts the value of \texttt{rs\textsubscript{0}} right by \texttt{rs\textsubscript{1}} bits.

\vspace{3ex}

\textbf{Action}
\begin{lstlisting}[frame=single]
	rd := rs0 >> rs1
\end{lstlisting}

\vspace{3ex}

\textbf{Assembler Syntax}
\begin{lstlisting}[frame=single]
	BSR {rd}, {rs0}, {rs1}
\end{lstlisting}

\vspace{3ex}

\textbf{Example}
\begin{lstlisting}[frame=single]
	BSR r0, r1, r2
\end{lstlisting}
