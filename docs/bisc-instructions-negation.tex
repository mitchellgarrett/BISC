\subsection{Negation Instructions}

\subsubsection{NOT \normalfont{- Logical Not}}\label{sec:NOT}

\vspace{3ex}

\begin{center}
	\begin{bytefield}[leftcurly=., leftcurlyspace=0pt]{32}
		\bitheader[endianness=little]{0, 7, 8, 15, 16, 23, 24, 31} \\
		\begin{leftwordgroup}{\hyperref[sec:t-type]{\textbf{D-Type}}}
		\bitbox{8}{0x17} & 
		\bitbox{8}{rd} &
		\bitbox{8}{rs} &
		\bitbox{8}{\textit{null}}
		\end{leftwordgroup}\\
	\end{bytefield}
\end{center}

\textbf{Description}

The NOT instruction performs a logical not operation on the source register and stores the result in the destination register.
Practically, this means that \texttt{rd} will be 1 if \texttt{rs} is 0 and 0 if \texttt{rs} is non-zero.

\vspace{3ex}

\textbf{Action}
\begin{lstlisting}[frame=single]
	if rs = 0, rd := 1, else rd := 0
\end{lstlisting}

\vspace{3ex}

\textbf{Assembler Syntax}
\begin{lstlisting}[frame=single]
	NOT {rd}, {rs}
\end{lstlisting}

\vspace{3ex}

\textbf{Example}
\begin{lstlisting}[frame=single]
	; r0 = not r1
	MODU r0, r1
\end{lstlisting}

\subsubsection{NEG \normalfont{- Integer Negate}}\label{sec:NEG}
 
\vspace{3ex}

\begin{center}
	\begin{bytefield}[leftcurly=., leftcurlyspace=0pt]{32}
		\bitheader[endianness=little]{0, 7, 8, 15, 16, 23, 24, 31} \\
		\begin{leftwordgroup}{\hyperref[sec:t-type]{\textbf{D-Type}}}
		\bitbox{8}{0x18} & 
		\bitbox{8}{rd} &
		\bitbox{8}{rs} &
		\bitbox{8}{\textit{null}}
		\end{leftwordgroup}\\
	\end{bytefield}
\end{center}

\textbf{Description}

The NEG instruction negates the source register and stores the result in the destination register,
essentially multiplying \texttt{rs} by -1.

\vspace{3ex}

\textbf{Action}
\begin{lstlisting}[frame=single]
	rd := -1 * rs
\end{lstlisting}

\vspace{3ex}

\textbf{Assembler Syntax}
\begin{lstlisting}[frame=single]
	NEG {rd}, {rs}
\end{lstlisting}

\vspace{3ex}

\textbf{Example}
\begin{lstlisting}[frame=single]
	; r0 = -1 * r1
	NEG r0, r1
\end{lstlisting}

\subsubsection{INV \normalfont{- Bitwise Invert}}\label{sec:INV}

\vspace{3ex}

\begin{center}
	\begin{bytefield}[leftcurly=., leftcurlyspace=0pt]{32}
		\bitheader[endianness=little]{0, 7, 8, 15, 16, 23, 24, 31} \\
		\begin{leftwordgroup}{\hyperref[sec:t-type]{\textbf{D-Type}}}
		\bitbox{8}{0x19} & 
		\bitbox{8}{rd} &
		\bitbox{8}{rs} &
		\bitbox{8}{\textit{null}}
		\end{leftwordgroup}\\
	\end{bytefield}
\end{center}

\textbf{Description}

The INV instruction inverts all the bits in the source register and stores the result in the destination register.

\vspace{3ex}

\textbf{Action}
\begin{lstlisting}[frame=single]
	rd := rs $\oplus$ 0xffffffff
\end{lstlisting}

\vspace{3ex}

\textbf{Assembler Syntax}
\begin{lstlisting}[frame=single]
	INV {rd}, {rs}
\end{lstlisting}

\vspace{3ex}

\textbf{Example}
\begin{lstlisting}[frame=single]
	; r0 = invert all bits of r1
	INV r0, r1
\end{lstlisting}