\subsection{Arithmetic Instructions}

\subsubsection{ADD \normalfont{- Add}}\label{sec:ADD}

\vspace{3ex}

\begin{center}
	\begin{bytefield}[leftcurly=., leftcurlyspace=0pt]{32}
		\bitheader[endianness=little]{0, 7, 8, 15, 16, 23, 24, 31} \\
		\begin{leftwordgroup}{\hyperref[sec:t-type]{\textbf{T-Type}}}
		\bitbox{8}{0x0F} & 
		\bitbox{8}{rd} &
		\bitbox{8}{rs\textsubscript{0}}
		\bitbox{8}{rs\textsubscript{1}}
		\end{leftwordgroup}\\
	\end{bytefield}
\end{center}

\textbf{Description}

The ADD instruction adds the values of the two source registers and stores the result in the destination register.
Note that the addition is the same whether the inputs are interpreted as signed or unsigned values.
The destination register can be the same as one of the source registers to overwrite that register with the resulting sum,
or all three registers can be the same to add a register to itself.

\vspace{3ex}

\textbf{Action}
\begin{lstlisting}[frame=single]
	rd := rs0 + rs1
\end{lstlisting}

\vspace{3ex}

\textbf{Assembler Syntax}
\begin{lstlisting}[frame=single]
	ADD {rd}, {rs0} {rs1}
\end{lstlisting}

\vspace{3ex}

\textbf{Example}
\begin{lstlisting}[frame=single]
	; r0 = r1 + r2
	ADD r0, r1, r2
	
	; r0 = r1 + r0
	ADD r0, r1, r0
	
	; r0 = r0 + r0
	ADD r0, r0, r0
\end{lstlisting}

\subsubsection{SUB \normalfont{- Subtract}}\label{sec:SUB}

\vspace{3ex}

\begin{center}
	\begin{bytefield}[leftcurly=., leftcurlyspace=0pt]{32}
		\bitheader[endianness=little]{0, 7, 8, 15, 16, 23, 24, 31} \\
		\begin{leftwordgroup}{\hyperref[sec:t-type]{\textbf{T-Type}}}
		\bitbox{8}{0x0E} & 
		\bitbox{8}{rd} &
		\bitbox{8}{rs\textsubscript{0}}
		\bitbox{8}{rs\textsubscript{1}}
		\end{leftwordgroup}\\
	\end{bytefield}
\end{center}

\textbf{Description}

The SUB instruction subtracts the value of \texttt{rs\textsubscript{1}} from \texttt{rs\textsubscript{0}} and stores the result in the destination register.
Note that the subtraction is the same whether the inputs are interpreted as signed or unsigned values.
The destination register can be the same as one of the source registers to overwrite that register with the resulting difference,
or all three registers can be the same to subtract a register from itself, but that would result in 0.
\vspace{3ex}

\textbf{Action}
\begin{lstlisting}[frame=single]
	rd := rs0 - rs1
\end{lstlisting}

\vspace{3ex}

\textbf{Assembler Syntax}
\begin{lstlisting}[frame=single]
	SUB {rd}, {rs0} {rs1}
\end{lstlisting}

\vspace{3ex}

\textbf{Example}
\begin{lstlisting}[frame=single]
	; r0 = r1 - r2
	SUB r0, r1, r2
	
	; r0 = r1 - r0
	SUB r0, r1, r0
\end{lstlisting} 

\subsubsection{MUL \normalfont{- Multiply}}\label{sec:MUL}

\subsubsection{MULH \normalfont{- Multiply High}}\label{sec:MULH}

\subsubsection{MULHU \normalfont{- Multiply High Unsigned}}\label{sec:MULHU}

\subsubsection{DIV \normalfont{- Divide}}\label{sec:DIV}

\subsubsection{DIVU \normalfont{- Divide Unsigned}}\label{sec:DIVU}

\subsubsection{MOD \normalfont{- Modulo}}\label{sec:MOD}

\subsubsection{MODU \normalfont{- Modulo Unsigned}}\label{sec:MODU}