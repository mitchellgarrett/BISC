\subsection{Arithmetic Instructions}

\subsubsection{ADD \normalfont{- Add}}\label{sec:ADD}

\vspace{3ex}

\begin{center}
	\begin{bytefield}[leftcurly=., leftcurlyspace=0pt]{32}
		\bitheader[endianness=little]{0, 7, 8, 15, 16, 23, 24, 31} \\
		\begin{leftwordgroup}{\hyperref[sec:t-type]{\textbf{T-Type}}}
		\bitbox{8}{0x0F} & 
		\bitbox{8}{rd} &
		\bitbox{8}{rs\textsubscript{0}}
		\bitbox{8}{rs\textsubscript{1}}
		\end{leftwordgroup}\\
	\end{bytefield}
\end{center}

\textbf{Description}

The ADD instruction adds the values of the two source registers and stores the result in the destination register.
Note that the addition is the same whether the inputs are interpreted as signed or unsigned values.
The destination register can be the same as one of the source registers to overwrite that register with the resulting sum,
or all three registers can be the same to add a register to itself.

\vspace{3ex}

\textbf{Action}
\begin{lstlisting}[frame=single]
	rd := rs0 + rs1
\end{lstlisting}

\vspace{3ex}

\textbf{Assembler Syntax}
\begin{lstlisting}[frame=single]
	ADD {rd}, {rs0} {rs1}
\end{lstlisting}

\vspace{3ex}

\textbf{Example}
\begin{lstlisting}[frame=single]
	; r0 = r1 + r2
	ADD r0, r1, r2
	
	; r0 = r1 + r0
	ADD r0, r1, r0
	
	; r0 = r0 + r0
	ADD r0, r0, r0
\end{lstlisting}

\subsubsection{SUB \normalfont{- Subtract}}\label{sec:SUB}

\vspace{3ex}

\begin{center}
	\begin{bytefield}[leftcurly=., leftcurlyspace=0pt]{32}
		\bitheader[endianness=little]{0, 7, 8, 15, 16, 23, 24, 31} \\
		\begin{leftwordgroup}{\hyperref[sec:t-type]{\textbf{T-Type}}}
		\bitbox{8}{0x0E} & 
		\bitbox{8}{rd} &
		\bitbox{8}{rs\textsubscript{0}}
		\bitbox{8}{rs\textsubscript{1}}
		\end{leftwordgroup}\\
	\end{bytefield}
\end{center}

\textbf{Description}

The SUB instruction subtracts the value of \texttt{rs\textsubscript{1}} from \texttt{rs\textsubscript{0}} and stores the result in the destination register.
Note that the subtraction is the same whether the inputs are interpreted as signed or unsigned values.
The destination register can be the same as one of the source registers to overwrite that register with the resulting difference,
or all three registers can be the same to subtract a register from itself, but that would result in 0.
\vspace{3ex}

\textbf{Action}
\begin{lstlisting}[frame=single]
	rd := rs0 - rs1
\end{lstlisting}

\vspace{3ex}

\textbf{Assembler Syntax}
\begin{lstlisting}[frame=single]
	SUB {rd}, {rs0} {rs1}
\end{lstlisting}

\vspace{3ex}

\textbf{Example}
\begin{lstlisting}[frame=single]
	; r0 = r1 - r2
	SUB r0, r1, r2
	
	; r0 = r1 - r0
	SUB r0, r1, r0
\end{lstlisting} 

\subsubsection{MUL \normalfont{- Multiply}}\label{sec:MUL}

\vspace{3ex}

\begin{center}
	\begin{bytefield}[leftcurly=., leftcurlyspace=0pt]{32}
		\bitheader[endianness=little]{0, 7, 8, 15, 16, 23, 24, 31} \\
		\begin{leftwordgroup}{\hyperref[sec:t-type]{\textbf{T-Type}}}
		\bitbox{8}{0x10} & 
		\bitbox{8}{rd} &
		\bitbox{8}{rs\textsubscript{0}}
		\bitbox{8}{rs\textsubscript{1}}
		\end{leftwordgroup}\\
	\end{bytefield}
\end{center}

\textbf{Description}

The MUL instruction multiplies the values of \texttt{rs\textsubscript{1}} and \texttt{rs\textsubscript{0}} and stores the product in the destination register.
Note that MUL multiplies the lower 16 bits of the source register, so the result is the same whether the inputs are interpreted as signed or unsigned values.
16-bit multiplication results in a 32-bit value so the entirety of the destination register will be used.
The destination register can be the same as one of the source registers to overwrite that register with the resulting product,
or all three registers can be the same to multiply a register by itself.
\vspace{3ex}

\textbf{Action}
\begin{lstlisting}[frame=single]
	rd := rs0[0..15] $\cdot$ rs1[0..15]
\end{lstlisting}

\vspace{3ex}

\textbf{Assembler Syntax}
\begin{lstlisting}[frame=single]
	MUL {rd}, {rs0} {rs1}
\end{lstlisting}

\vspace{3ex}

\textbf{Example}
\begin{lstlisting}[frame=single]
	; r0 = r1 * r2
	MUL r0, r1, r2
	
	; r0 = r1 * r0
	MUL r0, r1, r0
\end{lstlisting} 

\subsubsection{MULH \normalfont{- Multiply High}}\label{sec:MULH}

\vspace{3ex}

\begin{center}
	\begin{bytefield}[leftcurly=., leftcurlyspace=0pt]{32}
		\bitheader[endianness=little]{0, 7, 8, 15, 16, 23, 24, 31} \\
		\begin{leftwordgroup}{\hyperref[sec:t-type]{\textbf{T-Type}}}
		\bitbox{8}{0x11} & 
		\bitbox{8}{rd} &
		\bitbox{8}{rs\textsubscript{0}}
		\bitbox{8}{rs\textsubscript{1}}
		\end{leftwordgroup}\\
	\end{bytefield}
\end{center}

\textbf{Description}

The MULH instruction multiplies the upper 16 bits of \texttt{rs\textsubscript{1}} and \texttt{rs\textsubscript{0}} and stores the product in the destination register.
The upper 16 bits are interpreted as signed values, resulting in a different product than MULHU.
16-bit multiplication results in a 32-bit value so the entirety of the destination register will be used.
The destination register can be the same as one of the source registers to overwrite that register with the resulting product,
or all three registers can be the same to multiply a register by itself.
\vspace{3ex}

\textbf{Action}
\begin{lstlisting}[frame=single]
	rd := rs0[16..31] $\cdot$ rs1[16..31]
\end{lstlisting}

\vspace{3ex}

\textbf{Assembler Syntax}
\begin{lstlisting}[frame=single]
	MULH {rd}, {rs0} {rs1}
\end{lstlisting}

\vspace{3ex}

\textbf{Example}
\begin{lstlisting}[frame=single]
	; r0 = r1 * r2
	MULH r0, r1, r2
	
	; r0 = r1 * r0
	MULH r0, r1, r0
\end{lstlisting} 

\subsubsection{MULHU \normalfont{- Multiply High Unsigned}}\label{sec:MULHU}

\vspace{3ex}

\begin{center}
	\begin{bytefield}[leftcurly=., leftcurlyspace=0pt]{32}
		\bitheader[endianness=little]{0, 7, 8, 15, 16, 23, 24, 31} \\
		\begin{leftwordgroup}{\hyperref[sec:t-type]{\textbf{T-Type}}}
		\bitbox{8}{0x12} & 
		\bitbox{8}{rd} &
		\bitbox{8}{rs\textsubscript{0}}
		\bitbox{8}{rs\textsubscript{1}}
		\end{leftwordgroup}\\
	\end{bytefield}
\end{center}

\textbf{Description}

The MULHU instruction multiplies the upper 16 bits of \texttt{rs\textsubscript{1}} and \texttt{rs\textsubscript{0}} and stores the product in the destination register.
The upper 16 bits are interpreted as unsigned values, resulting in a different product than MULH.
16-bit multiplication results in a 32-bit value so the entirety of the destination register will be used.
The destination register can be the same as one of the source registers to overwrite that register with the resulting product,
or all three registers can be the same to multiply a register by itself.
\vspace{3ex}

\textbf{Action}
\begin{lstlisting}[frame=single]
	rd := rs0[16..31] $\cdot$ rs1[16..31]
\end{lstlisting}

\vspace{3ex}

\textbf{Assembler Syntax}
\begin{lstlisting}[frame=single]
	MULHU {rd}, {rs0} {rs1}
\end{lstlisting}

\vspace{3ex}

\textbf{Example}
\begin{lstlisting}[frame=single]
	; r0 = r1 * r2
	MULHU r0, r1, r2
	
	; r0 = r1 * r0
	MULHU r0, r1, r0
\end{lstlisting}

\subsubsection{DIV \normalfont{- Divide}}\label{sec:DIV}

\vspace{3ex}

\begin{center}
	\begin{bytefield}[leftcurly=., leftcurlyspace=0pt]{32}
		\bitheader[endianness=little]{0, 7, 8, 15, 16, 23, 24, 31} \\
		\begin{leftwordgroup}{\hyperref[sec:t-type]{\textbf{T-Type}}}
		\bitbox{8}{0x13} & 
		\bitbox{8}{rd} &
		\bitbox{8}{rs\textsubscript{0}}
		\bitbox{8}{rs\textsubscript{1}}
		\end{leftwordgroup}\\
	\end{bytefield}
\end{center}

\textbf{Description}

The DIV instruction divides \texttt{rs\textsubscript{1}} by \texttt{rs\textsubscript{0}} and stores the result in the destination register.
The source registers are interpreted as signed values, resulting in a different product than DIVU.
The destination register can be the same as one of the source registers to overwrite that register with the resulting quotient.
\vspace{3ex}

\textbf{Action}
\begin{lstlisting}[frame=single]
	rd := rs0 $\div$ rs1
\end{lstlisting}

\vspace{3ex}

\textbf{Assembler Syntax}
\begin{lstlisting}[frame=single]
	DIV {rd}, {rs0} {rs1}
\end{lstlisting}

\vspace{3ex}

\textbf{Example}
\begin{lstlisting}[frame=single]
	; r0 = r1 / r2
	DIV r0, r1, r2
	
	; r0 = r1 / r0
	DIV r0, r1, r0
\end{lstlisting}

\subsubsection{DIVU \normalfont{- Divide Unsigned}}\label{sec:DIVU}

\vspace{3ex}

\begin{center}
	\begin{bytefield}[leftcurly=., leftcurlyspace=0pt]{32}
		\bitheader[endianness=little]{0, 7, 8, 15, 16, 23, 24, 31} \\
		\begin{leftwordgroup}{\hyperref[sec:t-type]{\textbf{T-Type}}}
		\bitbox{8}{0x14} & 
		\bitbox{8}{rd} &
		\bitbox{8}{rs\textsubscript{0}}
		\bitbox{8}{rs\textsubscript{1}}
		\end{leftwordgroup}\\
	\end{bytefield}
\end{center}

\textbf{Description}

The DIVU instruction divides \texttt{rs\textsubscript{1}} by \texttt{rs\textsubscript{0}} and stores the result in the destination register.
The source registers are interpreted as unsigned values, resulting in a different product than DIV.
The destination register can be the same as one of the source registers to overwrite that register with the resulting quotient.
\vspace{3ex}

\textbf{Action}
\begin{lstlisting}[frame=single]
	rd := rs0 $\div$ rs1
\end{lstlisting}

\vspace{3ex}

\textbf{Assembler Syntax}
\begin{lstlisting}[frame=single]
	DIVU {rd}, {rs0} {rs1}
\end{lstlisting}

\vspace{3ex}

\textbf{Example}
\begin{lstlisting}[frame=single]
	; r0 = r1 / r2
	DIVU r0, r1, r2
	
	; r0 = r1 / r0
	DIVU r0, r1, r0
\end{lstlisting}

\subsubsection{MOD \normalfont{- Modulo}}\label{sec:MOD}

\vspace{3ex}

\begin{center}
	\begin{bytefield}[leftcurly=., leftcurlyspace=0pt]{32}
		\bitheader[endianness=little]{0, 7, 8, 15, 16, 23, 24, 31} \\
		\begin{leftwordgroup}{\hyperref[sec:t-type]{\textbf{T-Type}}}
		\bitbox{8}{0x15} & 
		\bitbox{8}{rd} &
		\bitbox{8}{rs\textsubscript{0}}
		\bitbox{8}{rs\textsubscript{1}}
		\end{leftwordgroup}\\
	\end{bytefield}
\end{center}

\textbf{Description}

The MOD instruction conducts a modulus operation and divides \texttt{rs\textsubscript{1}} by \texttt{rs\textsubscript{0}} and stores the remainder of that division in the destination register.
The source registers are interpreted as signed values, resulting in a different product than MODU.

\vspace{3ex}

\textbf{Action}
\begin{lstlisting}[frame=single]
	rd := rs0 % rs1
\end{lstlisting}

\vspace{3ex}

\textbf{Assembler Syntax}
\begin{lstlisting}[frame=single]
	MOD {rd}, {rs0} {rs1}
\end{lstlisting}

\vspace{3ex}

\textbf{Example}
\begin{lstlisting}[frame=single]
	; r0 = r1 % r2
	MOD r0, r1, r2
	
	; r0 = r1 % r0
	MOD r0, r1, r0
\end{lstlisting}

\subsubsection{MODU \normalfont{- Modulo Unsigned}}\label{sec:MODU}

\vspace{3ex}

\begin{center}
	\begin{bytefield}[leftcurly=., leftcurlyspace=0pt]{32}
		\bitheader[endianness=little]{0, 7, 8, 15, 16, 23, 24, 31} \\
		\begin{leftwordgroup}{\hyperref[sec:t-type]{\textbf{T-Type}}}
		\bitbox{8}{0x16} & 
		\bitbox{8}{rd} &
		\bitbox{8}{rs\textsubscript{0}}
		\bitbox{8}{rs\textsubscript{1}}
		\end{leftwordgroup}\\
	\end{bytefield}
\end{center}

\textbf{Description}

The MODU instruction conducts a modulus operation and divides \texttt{rs\textsubscript{1}} by \texttt{rs\textsubscript{0}} and stores the remainder of that division in the destination register.
The source registers are interpreted as unsigned values, resulting in a different product than MOD.

\vspace{3ex}

\textbf{Action}
\begin{lstlisting}[frame=single]
	rd := rs0 % rs1
\end{lstlisting}

\vspace{3ex}

\textbf{Assembler Syntax}
\begin{lstlisting}[frame=single]
	MODU {rd}, {rs0} {rs1}
\end{lstlisting}

\vspace{3ex}

\textbf{Example}
\begin{lstlisting}[frame=single]
	; r0 = r1 % r2
	MODU r0, r1, r2
	
	; r0 = r1 % r0
	MODU r0, r1, r0
\end{lstlisting}
