\section{Registers}

\subsection{Register Overview}

\begin{center}
	\begin{tabular}{|c|c|c|}
		\hline
		\textbf{Index} & \textbf{Register} & \textbf{Description} \\
		\hline
		\textit{0x00} & pc & Program Counter \\
		\hline
		\textit{0x01} & sp & Stack Pointer \\
		\hline
		\textit{0x02} & gp & Global Pointer \\
		\hline
		\textit{0x03} & fp & Frame Pointer \\
		\hline
		\textit{0x04} & ra & Return Address \\
		\hline
		\textit{0x05} & rv & Return Value \\
		\hline
		\textit{0x06} & ti & Temporary Immediate \textit{(Used by assembler)} \\
		\hline
		\textit{0x07} & ta & Temporary Address \textit{(Used by assembler)} \\
		\hline
		\textit{0x08-0x0F} & r0-r7 & General Purpose Registers \\
		\hline
		\textit{0x10-0x17} & f0-f7 & General Purpose Floating Point Registers \\
		\hline
	\end{tabular}
\end{center}
	
\subsection{Register Descriptions}

\subsubsection{Program Counter}
The Program Counter, or \textit{pc}, is a 32-bit register that stores the memory address, in bytes, of the next instruction the CPU will run.
The value of the Program Counter is changed by each instruction and is usually incremented by 4, however the $jump$ instructions may set its value to a specific address depending on the outcome of their condition.
The Program Counter should \textit{NEVER} be manually modified.

\subsubsection{Pointer Registers}
The Stack Pointer, \textit{sp}, Global Pointer, \textit{gp}, and Frame Pointer, \textit{fp}, are all registers used explicitly to point to addresses in memory.
The Stack Pointer points to the top of the $stack$ and is modified when pushing or popping values from the stack.
The Global Pointer is used to point to an arbitrary address in memory, but usually for things like global $variables$.
The Frame Pointer is similar to the Stack Pointer but is used specifically to point to the stack frame produced by a $function call$. 

\subsubsection{Return Address}
The Return Address register, \textit{ra}, stores the value of the Program Counter that a function should return to after it is finished.
The \textit{CALL} instruction sets \textit{ra := pc + 4}, then jumps to the address of the function being called.
When the function is finished, the \textit{RET} instruction sets \textit{pc := ra}, returning execution to the instruction right after the function call.

\subsubsection{Return Value}
The Return Value register, \textit{rv}, is used to store the return value of a function, if necessary.
It is the responsibility of the function to set the value of \textit{rv}.
If the function has no need to return any data, this register can be ignored.

\subsubsection{Temporary Registers}
The Temporary Immediate, \textit{ti}, and Temporary Address, \textit{ta}, registers are temporary registers used by the assembler for the processing of $pseudo-instructions$.
They can be used like any other general-purpose register, however that is not recommended as their values cannot be relied upon.

\subsubsection{General Purpose Registers}
Registers \textit{r0-r7} are general purpose registers, meaning that they can be used for whatever purpose deemed necessary, meaning that these are the registers that the programmer will use the most.